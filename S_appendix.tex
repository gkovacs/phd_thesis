\section{List of Browser Interventions}

% There are 27 interventions total: 7 generic interventions that can beused on all sites, 5 interventions designed specifically for Facebook, and additional ones designedspecifically for YouTube, Reddit, Twitter, Netflix, Gmail, Amazon, iQiyi, and Youku

The following is the list of interventions used for this study, showing the intervention name and description as seen by the end user.\\

Generic interventions that can be used on all sites:
\begin{small}
\begin{itemize}
    \item Minute Watch: Notifies you of time spent every minute
    \item Supervisor: Shows time spent on site at the top of screen
    \item Scroll Freezer: Freezes scrolling after a certain amount of scrolls
    \item Stat Whiz: Show time spent and visit count each visit
    \item GateKeeper: Makes you wait a few seconds before visiting
    \item 1Min Assassin: Closes tab after 60 seconds
    \item Bouncer: Asks how long you want to spend on site this visit
\end{itemize}
\end{small}
\vspace{2mm}

Facebook-specific interventions:

\begin{itemize}
    \item Time Injector: Injects timer into the Facebook feed
    \item Feed Eater: Removes the Facebook news feed
    \item TimeKeeper: Notifies you of time spent in the corner of your desktop
    \item No Comment: Removes Facebook comments
    \item Clickbait Mosaic: Removes clickbait from the news feed
\end{itemize}

\vspace{2mm}

Youtube-specific interventions:

\begin{itemize}
    \item Sidekicker: Remove sidebar links
    \item Think Twice: Prompt the user before watching a video
    \item No Comment: Removes comment section
\end{itemize}

\vspace{2mm}

Netflix-specific interventions:

\begin{itemize}
    \item Fun Facts: Gives you a fact and links an article on the effect of TV
    \item Alarm Clock: Asks the user to set an alarm before watching a show
    \item Stop Autoplay: Stops the site from automatically playing the next video
\end{itemize}

\vspace{2mm}

Reddit-specific interventions:

\begin{itemize}
    \item Comment Remover: Removes Reddit comments
    \item Mission Objective: Asks what you aim to do this visit and puts a reminder up
\end{itemize}

\vspace{2mm}

Youku-specific interventions

\begin{itemize}
    \item Think Twice: Prompt the user before watching a video
    \item Sidekicker: Remove sidebar links
\end{itemize}

\vspace{2mm}

iQiyi-specific interventions

\begin{itemize}
    \item Think Twice: Prompt the user before watching a video
    \item Sidekicker: Remove sidebar links
\end{itemize}

\vspace{2mm}

Twitter-specific interventions:

\begin{itemize}
    \item Feed Eater: Removes the Twitter news feed
\end{itemize}

\vspace{2mm}

Amazon-specific interventions:

\begin{itemize}
    \item No Recs: Hides recommendations
\end{itemize}

\vspace{2mm}

Gmail-specific interventions

\begin{itemize}
    \item Speedbump: Delays the arrival of new emails
\end{itemize}

\vspace{3mm}


\section{List of Mobile Interventions}

All mobile interventions are generic, that is they can be used on any app.

\begin{small}

\begin{itemize}
    \item At it Again: Sends a pop up with your app visit count.
    \item Progress Report: Sends a pop up with today's total usage for a certain app
    \item Red Alert!: Sends a notification with today's total usage for a certain app
    \item Repeat Offender: Sends a notification with your app visit count
    \item All in All: Pops a dialog with the day's total time on the current app
    \item Back To Target: Suggests you to visit a target app
    \item Counting on You: Puts a timer on screen in watchlisted apps
    \item Man Overboard! Shows a dialog with your app visit count
    \item No Peeking!: Asks for confirmation before opening watchlisted apps
    \item Wait Up! Pause for 10 seconds before entering an app
    \item Your Better Half: Sends a pop up to go to a target app
    \item Look on the Bright Side: Dim the screen a little at a time
    \item Take Your Pick: Select how long you want to spend on an app
    \item The Final Countdown: On screen timer that closes the app when time runs out
\end{itemize}

\end{small}


The following interventions apply across the device as a whole, not individual applications.

\begin{small}
\begin{itemize}
    \item How Time Flies!: Sends a pop up message with current app visit length
    \item Knock Knock: Sends a pop up with your glance count for the day
    \item Long Time No See: Sends pop up with your phone usage for the day
    \item Call it a Day: Sends notification with phone usage for the day
    \item Easy on the Eyes: Sends notification with glance count for the day
    \item Hello, Old Friend: Sends notification with unlock count for the day
    \item The Clock is Ticking: Sends a notification with the current app visit duration
    \item En Garde: Pops a dialog with the day's total unlock count
    \item Hold the Phone: Show dialog with phone usage for the day
    \item Long Story Short: Pops a dialog with the visit time for the current app
    \item Quote reminder: Show quote upon opening app
    \item Time Reminder: Show dialog with phone usage for the day
    \item Take Your Pick: Select how long you want to spend on an app
\end{itemize}
\end{small}

\section{User Feedback}

User feedback for HabitLab has been generally positive. On the Chrome store for the browser version, there are 26 reviews, with an average rating of 4.5 stars, while on the Play store for the mobile version, there are 24 reviews, with an average rating of 4 stars. Users leave us feedback, both positive and negative, in a number of forms -- through feedback forms within the interface, by filing issues on GitHub, or sending emails.

(Types of feedback and examples of them)

We find that some user feedback often request a specific intervention. A commonly requested feature is the ability to have multiple interventions active at once.  TODO

\msb{What kind of user feedback have we gotten?}

\msb{Would love to see a page of Discussion in this chapter. Possible topics include: (1) Do you think the same design principles would allow a research system to succeed in other domains? If so, why? If not, why not? (2) What did we try that didn't work, and why? (3) Of the stuff that you did in designing HabitLab, what do you think is the most central novel idea?}