\chapter{Changes in Preferences over Time}

In order to be effective at helping users achieve their goals, behavior change must be aware of not just the goals themselves, but how motivated users are to achieve them. This chapter will discuss a set of studies analyzing the goals of users using HabitLab, and an analysis of how their motivation levels change over time.

Users are diverse and have differing objectives for using software. Even in systems such as HabitLab where explicitly state their goals during onboarding, as wanting to spend less time on the sites that they select, the degree and means by which they wish to achieve this reduction can differ. We observed evidence of this in our uninstallation survey, where some users indicated they were uninstalling because interventions were too easy, while others were uninstalling because interventions were too difficult.

%In this chapter, we will first analyze how 

\section{Introduction}

Getting the right level of challenge and difficulty in behavior change systems is a challenging question. These systems must decide how aggressive they are in pushing users towards the desired behavior. A mismatch between the user's desired intervention difficulty, and the intervention difficulty selected by the system, may lead to the system being less effective, or being uninstalled. This leads to a design question of how do we choose the appropriate difficulty for the user to trade off attrition versus behavior change outcomes.

A number of options are available in choosing difficulty. One is to allow users to choose their preferred difficulty. However, this suffers from the possibility that their preferences may change over time. Another is to assign defaults that will push them towards the desired difficulty. Yet another is to intelligently predict the optimal difficulty levels.

Choice architecture theories predict that defaults are beneficial .... user motivation declines .... users are initially overly optimistic

In our first study, we test whether user control is beneficial. Method results. What do they pick. We find they are overly optimistic because over time their choices fall.

In our second study we investigate the effects of random assignment and disregarding user choices.

In our third study we investigate how well we can predict the choices of user difficulty over time, and how frequently we need to ask to get a accurate prediction.