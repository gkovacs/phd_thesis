\chapter{Conclusion}

In this thesis we have proposed a paradigm of in-the-wild experimentation to gain insights about behavior change, and have created a platform, HabitLab, to realize this vision in the context of helping users reduce their time online and on their phones. We also conducted a set of studies on HabitLab which illustrate that we can make novel findings with the system.

The first set of studies we ran with HabitLab investigated whether interventions decline in effectiveness over time. We found that interventions decline in effectiveness if the same intervention is repeatedly shown, and that a strategy of rotating between different interventions can help improve the effectiveness. While this comes at the cost of increased attrition, most likely due to users having incorrect mental models, we can reduce this attrition via a simple design shown when a new intervention is introduced.

The second set of studies investigated whether interventions that help save time on one site, app, or device influence time spent elsewhere. We found that on the browser, reducing time on one site has a beneficial side effect of reducing time elsewhere. We believe this is due to reducing time on aggregator sites that drive traffic to other sites. On phones, however, we did not observe any side effects of reducing time on one app on other apps. We also did not observe any cross-device effects.

There is a large opportunity for behavior change research through big data and crowdsourcing that has been under-explored due to the paucity of large-scale deployments of research systems. Could we predict which interventions will work well for a new user, before they even start using the system? Could we automatically deploy and test modified versions of interventions, to hill-climb our way to more effective ones? Could we enlist an engaged user community to come up with, generate, and test new interventions for the long-tail of behavior change goals that designers had never even thought of? These can be realized with machine learning and crowdsourcing techniques, but there have not been appropriate communities for an in-the-wild deployment. We hope HabitLab will provide such a platform to realize this vision of community-driven behavior change research  in the wild. % through self-experimentation.
