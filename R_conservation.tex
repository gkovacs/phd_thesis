%----------------------------
% RQ
%\section{Behavior Change and Motivation}

%\msb{Take a pass through this section and remove phrases like "Studies show X" or "Researchers have found X". It's weak writing. Instead just state the claim X. Like instead of "Researchers have found that hotdogs are tasty [cite].", just write "Hotdogs are tasty [cite]." I removed several already.}

% The field of persuasive technology studies how technology can be used to influence behavior~\cite{fogg2002persuasive}. Persuasive technology systems have been successful in promoting behaviors such as sustainable resource usage~\cite{froehlich2009ubigreen}, fitness~\cite{consolvo2008activity}, sleep~\cite{kay2012lullaby,choe2011opportunities}, healthy eating~\cite{noronha2011platemate,epstein2016crumbs}, stress management~\cite{adams2014towards,sanches2010mind}, smoking cessation~\cite{paay2015personal}, and productivity~\cite{whittaker2016don, kim2016timeaware}. An influential framework of behavior change is the B=MAT model~\cite{fogg2002persuasive}, which states that desired behaviors result when motivation, ability, and a trigger (a call to action) are all present. Another framework of habit change is the habit loop~\cite{eyal2014hooked}, which tells us that designs can build habits via a repeated process of displaying a trigger, having the user take an action, providing a reward, and having the user invest in the system.

%\msb{what is lifestyle? it looks like sleep? is this capturing something else? I don't think lifestyle is the right word here}


%\msb{subsection titles for single paragraphs are unnecessary in related work; I'm cutting them}

\section{Measuring Overall Effectiveness}

% Persuasive technologies on computers and mobile phones can help support behavior change. Researchers have developed web-based systems to promote a variety of behavior change goals, including psycho-therapeutic interventions~\cite{doi:10.1080/15228830802094429}, promoting healthy habits~\cite{cugelman2013gamification, lyons2014behavior} and improving educational engagement~\cite{anderson2013steering, anderson2014engaging}. In parallel, a number of studies focus on mobile-based interventions~\cite{paredes2014poptherapy, doi:10.1111/j.1740-9713.2015.00863.x, RILEY201567, anderson2014engaging, FJELDSOE2009165, Whittaker09}. For example, PopTherapy studied micro-interventions for coping with stress using mobile devices~\cite{paredes2014poptherapy}. Likewise, HeartSteps developed an application on mobile phones to promote promote physical activities for individuals~\cite{doi:10.1111/j.1740-9713.2015.00863.x}. % For HabitLab, we developed a version for computers as well as a version for mobile phones. Our study studies user behaviors by evaluating real-time data from both platforms.

% These persuasive technologies have produced measurable beneficial effects. Web-based systems promote a behavior change goals including classroom engagement~\cite{anderson2014engaging, anderson2013steering}, psychology therapy~\cite{doi:10.1080/15228830802094429} and heathy habits~\cite{cugelman2013gamification, lyons2014behavior}. In parallel, a number of studies focused on mobile-based interventions~\cite{paredes2014poptherapy, RILEY201567, FJELDSOE2009165, Whittaker09, info:doi/10.2196/mhealth.4160}. \msb{I don't understand how this paragraph is different from the prior one. Didn't the prior one just list a bunch of domains for behavior change systems?} For instance, MyBehavior, an app on mobile phone, was built to track physical activities of the users and to provide personalized suggestions that are tailored to the users historical behavioral data~\cite{info:doi/10.2196/mhealth.4160}. Similarly, PopTherapy is an a mobile phone application that studied micro-interventions for coping with stress~\cite{paredes2014poptherapy}.

% Measuring the effectiveness of a persuasive system remains a major challenge in behavior change systems design. While behavior change systems can be effective during the experiments~\cite{doi:10.1080/15228830802094429, Cuijpers2008, info:doi/10.2196/jmir.1376}, many review papers are more restrained in whether behavior change systems remain effective outside the studies~\cite{doi:10.1111/j.1467-789X.2009.00646.x, 10.1371/journal.pmed.1000387, NORMAN2007336, 10.1007/978-3-319-07127-5_11}. The critique holds that behavior changes are long, complex processes, and the effectiveness of a system is hard to measure in a short period of time~\cite{prochaska1997transtheoretical}. For instance, an intervention promoting healthy habits, which is effective in changing participants eating habits, might reduce their physical activities which are not measured in the experiment~\cite{COTTER2014243}. Likewise, a system promoting physical activities might have no knowledge on participants' eating habits~\cite{doi:10.1111/j.1740-9713.2015.00863.x}. With interventions on both the computer and the mobile device for each user, our study draws on lived informatics by evaluating the efficacy of productivity interventions in the context of a more complete ecosystem which includes both computers and mobile devices.

Measuring the effectiveness of a persuasive system remains a major challenge in the design of behavior change systems. While behavior change systems can be effective during experiments~\cite{doi:10.1080/15228830802094429, Cuijpers2008, info:doi/10.2196/jmir.1376}, many review papers are more restrained in whether behavior change systems remain effective outside studies and bring longitudinal behavioral change~\cite{doi:10.1111/j.1467-789X.2009.00646.x, 10.1371/journal.pmed.1000387, NORMAN2007336, 10.1007/978-3-319-07127-5_11}. Because behavior changes are long and complex processes, the efficacy of a persuasive system is often difficult to measure~\cite{prochaska1997transtheoretical}. For instance, an intervention promoting healthy habits, which was effective in changing participants' eating habits, might reduce their physical activities, which were not measured in the experiment~\cite{COTTER2014243}. Likewise, a system promoting increased physical activity may be unable to observe effects on participants' eating habits~\cite{doi:10.1111/j.1740-9713.2015.00863.x}. Compared to prior work, our study examines these spillover effects in the context of a more complete ecosystem, including both desktop browsers and mobile devices. % \msb{I don't know what this means}.  \msb{use this paper in the intro to motivate the redistribution hypothesis}

\subsection{Multi-Device Usage}

Cyberslacking, referred to as non-work-related computing, is the use of Internet and mobile technology during work hours for personal purposes~\cite{VITAK20111751,PEE2008120, Pamela2004, lim2002}. One study found that employees spent at least one hour on non-work-related activities during a regular work day~\cite{VITAK20111751}. Researchers also reported that non-work-related Internet usage comprises approximately 30\%--50\% of total usage~\cite{RESTUBOG2011247,JAMALUDDIN2015495}. %  \msb{this sentence seems disconnected from everything that came before. why is it here?}.

%\subsection{The Vicious Cycle of Procrastination}

Unproductive time begets further unproductive time. For example, increased time spent online can increase sleep debt, which in turn leads to more time spent online \cite{Mark:2016:SDS:2858036.2858437}. Likewise, the Hook Model claims that many of the most addictive online sites use a cycle of investment techniques to keep users coming back---for example, making a post on Facebook may result in future notifications, which will in turn will get the user to come back and make more posts~\cite{eyal2014hooked}. Finally, sites such as Facebook, Reddit, Twitter, and Buzzfeed are filled with links to each others' content, so it may be the case that increasing usage of one will increase usage of others. If productivity interventions are able to break this vicious cycle of procrastination for one application, they may actually reduce time spent on other unproductive applications as well.

% The importance of understanding the effectiveness of productivity interventions in a complete ecosystem and the rising awareness of unproductive time spent on mobile devices call into focus: would productivity interventions reduce net unproductive time? Or is it a weak palliative with little discernible effect? This led to our research question:

%\begin{resques}[RQ]
%Do productivity interventions reduce net unproductive time, or just redistribute it to other applications, sites, and devices?
%Does rotating interventions is helpful in increasing effectiveness and lowering attrition in online behavior changes systems?}
%\end{resques}

\subsection{Distribution Of Unproductive Time}

%\msb{Same, remove "Studies show X" or "Researchers have found X". } \goli{done}

In this section, we will examine related studies in behavior change systems to develop testable hypotheses regarding the research question.

%\subsection{Intra-device}

Multitasking has become ubiquitous in today's workplaces~\cite{Appelbaum2016, mark2015multitasking, CARRIER2009483}. Multitasking is both essential and unavoidable in the workplace~\cite{freedman2007, mark2008cost}, and it takes 11 minutes on average before people switch to a new task~\cite{dabbish2011keep}. 
% The nature of multitasking in the workspace presents a challenge for designing effective productivity interventions in the face of multiple tasks. %concurrent tasks.
%Researchers also reported multi-tasking on computers is a required skill in employment recruiting nowadays. The study pointed out that an internet-based search of the terms "job description" and "multitasking", by using Google, had over 1.2 million results~\cite{Appelbaum2016}. 
% However, the nature of multitasking in the workspace increases the difficulty of designing effective productivity interventions when there are multiple concurrent tasks. % When one goal increases productivity via an intervention, people might redistribute their unproductive time to other goals instead. \msb{this claim is just thrown out without citations and dropped. either back it up or remove it}
% \msb{inconsistent throughout: is it multitasking or multi-tasking?}

% Persuasive systems often bring the intended behavioral changes (e.g.~\cite{doi:10.1080/15228830802094429,cugelman2013gamification, lyons2014behavior,anderson2013steering, anderson2014engaging}). These results suggest that, at the very least, HabitLab can reduce the time spent on unproductive activities. \msb{I don't see why this paragraph is relevant at all to developing H1 or H2. cut?}

% However, the time spent on unproductive activities might be redistributed elsewhere. While scholars found that persuasive technologies could produce positive outcomes, others worried that interventions could result in negative consequences which might promote users to adopt the opposite target behavior~\cite{10.1007/978-3-319-07127-5_11, 10.1007/978-3-319-31510-2_6}. For example, the tobacco industry sponsored anti-smoking campaign that supports parents to educate their children on risks of smoking could backfire, promoting some teens to light up~\cite{trove.nla.gov.au/work/31391712}. Similarly, researchers who study anti-bullying wristbands found that the wristband campaign increased the chances of bullying for the kids who wear it~\cite{antibullying}. \msb{No, these literatures don't suggest a redistribution hypothesis, it supports a sort of ``reverse effect'' hypothesis where the time on the targeted goal is increased rather than decreased}  This prompts our first hypothesis:
Studying behavior change effects across multiple devices is important: focusing on a single platform will myopically miss unproductive behaviors on other platforms. %because people distributed their unproductive behaviors across multiple People spend time on unproductive activities with computers and mobile phones. 
Attention is fragmented in both mobile and traditional desktop environments~\cite{socialmedia2010, mark2015multitasking}. The time spent on mobile devices has increased more rapidly than time on computers or TVs~\cite{multidevice, nielsen}. On the other hand, mobile applications have been regarded as substitutions of websites in many studies~\cite{10.1007/978-3-642-36516-4_7}. Large technology companies such as Facebook and Amazon have been focusing on user growth on mobile devices~\cite{socialmedia2010}.

However, interventions may result in unintended outcomes~\cite{trove.nla.gov.au/work/31391712, 10.1007/978-3-319-07127-5_11, 10.1007/978-3-319-31510-2_6}. Specifically, while some interventions may be highly effective at achieving the measured goal of a behavioral change system, they may reduce desired outcomes elsewhere~\cite{10.1007/978-3-319-07127-5_11}. In one health-related intervention, while the physical activity of participants increased, calorie intake also increased, working against the goal of promoting a healthy lifestyle~\cite{Blair1985}. Similarly, using peer pressure to build confidence for students at school would, in turn, lower their self-esteem which actually was opposite to the goal of augmenting confidence~\cite{10.1007/978-3-319-31510-2_6}.

% In our system, the time spent on unproductive activities might be decreased in one application yet increased in others. These prompt our hypotheses:
% \msb{didn't this come up in the previous section? move that down here, don't repeat yourself} \msb{runon sentence:}

%\begin{hyp}[H\ref*{hyp:within}] \label{hyp:within}
%Within a single device, productivity interventions will cause the time spent on targeted sites and apps to be redistributed to other sites and apps. 
%\end{hyp}

%\subsection{Inter-devices}

% Researchers found multi-tasking behaviors across different devices including computers and mobiles~\cite{Appelbaum2016}. Studies demonstrated that employees who spend time on non-work-related computing during working hours on both Internet and mobile technology~\cite{VITAK20111751}. \jake{I don't know what the preceeding sentence is trying to say. does spending time on non-related computing do something, or is this behavior simply observed?} Furthermore, some studies showed that the user growth rate is higher on mobile devices than on computers~\cite{multidevice}. \msb{Again, none of these results directly suggest this hypothesis.} In light of these results, we hypothesize:


%\msb{awkward phrasing:} \goli{changed first sentence!!}
% Mobile devices play an increasingly important role in our daily lives~\cite{multidevice} \msb{didn't you make this point in the previous section?}. Researchers found multitasking behaviors across different devices including mobile devices~\cite{Appelbaum2016}. Especially, people spent a significant portion of working hours on mobiles devices for unproductive activities~\cite{VITAK20111751}. \msb{all of these seem to echo the previous section?} As mobile phones become popular devices for unproductive activities in workspace, our interventions might increase productivity on computer while decrease on mobile devices. Similar to the intra-device's speculation, we hypothesize:

% \msb{that paragraph didn't do anything for me. I might be missing something, but my sense is that none of these papers make any kind of cross-platform redistribution prediction. if you don't have any evidence about cross-platform bleeding, then I would just go directly from H1 to H2, since the same causal pathway would lead to both}

%\begin{hyp}[H\ref*{hyp:across}] \label{hyp:across}
%Between computers and mobile devices, productivity interventions will cause the time spent on one device to be redistributed to other devices.
%\end{hyp}


% % The challenges of static interventions, and the rising wave of personalization systems, call into focus: would a rotation strategy work? Or is it a weak palliative with little discernible effect? This led to our research question:

% \begin{resques}[RQ]
% Can a strategy of rotating interventions produce more effective behavior change systems?
% %Does rotating interventions is helpful in increasing effectiveness and lowering attrition in online behavior changes systems?}
% \end{resques}

% \rev{One major topic inspiring our work is users' desires to curb or control their time spent on social media sites. People pressure themselves to, and often do, make efforts to reduce their time spent on social media sites such as Facebook and Twitter~\cite{Sleeper:2015:ILI:2675133.2675193,schoenebeck2014giving}. Yet this is difficult because users turn to social media to address their need to belong, the need for self-presentation, the need for self-esteem~\cite{nadkarni2012people}, the need for entertainment and gratification ~\cite{raacke2008myspace}, and self-affirmation ~\cite{toma2013self}. Whether social media use improves well-being is a complex question depending on the nature of the engagement ~\cite{uysal2013mediating, marche2012facebook, lin2015emotional, kim2011facebook, muise2009more, sagioglou2014facebook, tandoc2015facebook}, but thanks to instant gratification and sites' use of gamification~\cite{chou2015actionable, zichermann2011gamification, huotari2012defining} and behavior design techniques~\cite{fogg2002persuasive, eyal2014hooked} to drive engagement, users keep coming back, to the point that some consider it an addiction~\cite{andreassen2012development, ryan2014uses, tang2016personality, turel2014examination}.} %Furthermore, social media sites make heavy use of gamification to drive engagement on their sites~\cite{chou2015actionable, zichermann2011gamification, huotari2012defining}, and are cited as successful applications of behavior change theories~\cite{fogg2002persuasive, eyal2014hooked}.} %Furthermore, social media sites are designed to maximize user engagement, with features closely following gamification strategies~\cite{chou2015actionable} and behavior change theories~\cite{fogg2002persuasive, eyal2014hooked}.}


%----------------------------


%----------------------------
% RQ 
% \section{Behavior Change And Motivation}
% The field of persuasive technology studies how technology can be used to influence behavior~\cite{fogg2002persuasive}. Persuasive technology systems have been successful in promoting behaviors such as sustainable resource usage~\cite{froehlich2009ubigreen}, fitness~\cite{consolvo2008activity}, sleep~\cite{kay2012lullaby,choe2011opportunities}, healthy eating~\cite{noronha2011platemate,epstein2016crumbs}, stress management~\cite{adams2014towards,sanches2010mind}, smoking cessation~\cite{paay2015personal}, and productivity~\cite{whittaker2016don, kim2016timeaware}. An influential framework of behavior change is the B=MAT model~\cite{fogg2002persuasive}, which states that desired behaviors result when motivation, ability, and a trigger (a call to action) are all present. Another framework of habit change is the habit loop~\cite{eyal2014hooked}, which tells us that designs can build habits via a repeated process of displaying a trigger, having the user take an action, providing a reward, and having the user invest in the system.

% A number of taxonomies characterize the design space of interventions, both general~\cite{michie2013behavior, behaviourchangewheel, abraham2008taxonomy, dolanmindspace} and domain-specific~\cite{hardeman2000interventions, west2010behavior}. Michie's behavior change taxonomy lists 93 techniques for behavior change, clustered according to the cognitive phenomenon they target~\cite{michie2013behavior}. Systems have investigated effects of these techniques individually, such as using ``cheating'' to support lapse management~\cite{agapie2016staying}, using different framings to present results~\cite{kim2016timeaware}, or setting goals and plans~\cite{agapie2016plansourcing}.

% % One of the most comprehensive investigations we
% % identified is a recent meta-analysis of 85 studies by
% % Webb et al. (2010) that found interventions that were
% % strongly based in theory had greater impact than those
% % that were not, and interventions that incorporated more
% % behavior change techniques tended to have larger
% % effects than interventions that incorporated fewer techniques

% % Within the CSCW community, behavior change has been an active area of research. One major topic inspiring our work is users' desires to curb or control their time spent on social media sites. People pressure themselves to, and often do, make efforts to reduce their time spent on social media sites such as Facebook and Twitter~\cite{Sleeper:2015:ILI:2675133.2675193,schoenebeck2014giving}. This paper builds on this literature, contributing studies of how people might become more effective at this goal. Sociotechnical systems are also locations where people discuss behavior change~\cite{chancellornorms}, and find social support~\cite{Ko:2015:NGI:2675133.2675244, Chung:2017:PTB:3025453.3025747}. Behavior change often requires self-tracking, self-experimentation, and personal informatics~\cite{Karkar:2017:TFS:3025453.3025480}, leading to opportunities to share data and progress with trusted others~\cite{Chung:2016:BNA:2818048.2819926, pina2017personal}.

% \rev{People use a variety of sociotechnical systems to support behavior change, including forums~\cite{eysenbach2004health, chancellornorms}, social sharing~\cite{poirier2012social, Chung:2016:BNA:2818048.2819926, pina2017personal, Ko:2015:NGI:2675133.2675244}, personal informatics~\cite{li2010stage, Chung:2017:PTB:3025453.3025747}, and self-experimentation~\cite{Karkar:2017:TFS:3025453.3025480}. People use behavior change forums to gain social support~\cite{hong2012outcomes} -- meeting social needs such as approval and esteem~\cite{kaplan1977social}. They do so by providing users with information and advice~\cite{hong2012outcomes}, and establishing norms~\cite{chancellornorms}. They also facilitate social comparisons~\cite{davison2000talks} which influence behaviors, as social comparison theory states that users seek to bring their behaviors in line with norms~\cite{festinger1954theory}. Communities also help users find others with similar experiences~\cite{huh2014health} who can help them through the process of recovering and adapting to changes~\cite{newman2011s}. Social sharing~\cite{poirier2012social, richardson2010online} works by helping users receive support through social interactions, and encouraging accountability~\cite{epstein2015nobody}. Personal informatics support behavior change through stages of preparation, collection, integration, reflection, and action~\cite{li2010stage}. The theory of lived informatics~\cite{epstein2015lived} adds additional stages where users choose tracking tools, and alternate between lapsing and resuming their tracking behaviors. HabitLab combines personal informatics and self-experimentation to support behavior change. Our study draws on lived informatics by evaluating whether rotating interventions is an effective strategy to combat lapses such as ignoring interventions or uninstalling.}

% % Of course, sociotechnical systems may also have deleterious effects --- communities may set unhealthy norms such as encouraging eating disorders~\cite{chancellornorms}, and social sharing can cause anxiety~\cite{munson2010happier}, impression management concerns~\cite{consolvo2014designing}, and make users unwilling to set goals~\cite{munson2015effects}.

% % \msb{Can you pull this together into a summary of what we draw from this literature and how it influences HabitLab or our study?}

% \rev{One major topic inspiring our work is users' desires to curb or control their time spent on social media sites. People pressure themselves to, and often do, make efforts to reduce their time spent on social media sites such as Facebook and Twitter~\cite{Sleeper:2015:ILI:2675133.2675193,schoenebeck2014giving}. Yet this is difficult because users turn to social media to address their need to belong, the need for self-presentation, the need for self-esteem~\cite{nadkarni2012people}, the need for entertainment and gratification ~\cite{raacke2008myspace}, and self-affirmation ~\cite{toma2013self}. Whether social media use improves well-being is a complex question depending on the nature of the engagement ~\cite{uysal2013mediating, marche2012facebook, lin2015emotional, kim2011facebook, muise2009more, sagioglou2014facebook, tandoc2015facebook}, but thanks to instant gratification and sites' use of gamification~\cite{chou2015actionable, zichermann2011gamification, huotari2012defining} and behavior design techniques~\cite{fogg2002persuasive, eyal2014hooked} to drive engagement, users keep coming back, to the point that some consider it an addiction~\cite{andreassen2012development, ryan2014uses, tang2016personality, turel2014examination}.} %Furthermore, social media sites make heavy use of gamification to drive engagement on their sites~\cite{chou2015actionable, zichermann2011gamification, huotari2012defining}, and are cited as successful applications of behavior change theories~\cite{fogg2002persuasive, eyal2014hooked}.} %Furthermore, social media sites are designed to maximize user engagement, with features closely following gamification strategies~\cite{chou2015actionable} and behavior change theories~\cite{fogg2002persuasive, eyal2014hooked}.}

% % datu2012does

% % Social media is particularly addictive be

% % \rev{expanded the related work discussion on sociotechnical theories. list the theories driving these papers. social support cscw theories. why people use forums for behavior change support what role do other people play in driving your behavior change. collective exercise. why are social media particularly hard to manage your behavior with why is social media addictive.}

% % \rev{festinger's social comparison theory postulated that social behaviors could be predicted largely on the basis of the assumption that individuals seek to have and maintain a sense of normalcy and accuracy about the world}

% % http://journals.sagepub.com/doi/pdf/10.1177/1524839911405850
% % Harnessing Social Media for Health Promotion and Behavior Change

% % \rev{Communities and collaborative support can help achieve behavior change goals. }

% Much previous work has focused on gamification as an approach to design behavior change systems~\cite{deterding2011game}. Gamification has been shown to have positive effects on engagement and outcomes in behavior-change contexts such as promoting healthy habits~\cite{cugelman2013gamification, lyons2014behavior} and improving  educational engagement~\cite{anderson2013steering, anderson2014engaging}, though effectiveness varies depending on the context and design~\cite{6758978}. % Popular online services such as Facebook and LinkedIn make heavy use of gamification to drive engagement on their sites~\cite{chou2015actionable, zichermann2011gamification, huotari2012defining}.

% Attrition is a major challenge in behavior change systems. Attrition~\cite{eysenbach2005law}, also known as dropout, occurs when participants stop participating, leave, or uninstall the system. Persuasive systems built for weight control and therapy have shown substantial attrition rates in longitudinal studies~\cite{Bernier1986,paredes2014poptherapy}, and prior work in CSCW has sought to help reduce attrition rates through techniques drawn from dieting and addiction research~\cite{agapie2016staying}. % This paper contributes a direct study of attrition and its antecedents.%this study the increments in attrition has not been discussed in detail or quantified in the context of rotating interventions in behavior change systems.

% A recent trend in behavior change systems has been the concept of personalizing interventions. Such systems explore several possible strategies using techniques such as multi-armed bandits to find the intervention that is most effective for the user~\cite{paredes2014poptherapy, rabbi2014automated}. For example, PopTherapy demonstrated personalized messaging could be found through such techniques~\cite{paredes2014poptherapy}. Likewise, HeartSteps conducted tens or hundreds of micro-randomized trials on users~\cite{doi:10.1111/j.1740-9713.2015.00863.x}.  When multi-armed bandits are just beginning to get feedback from a user, they will try out several different interventions to see what works. This exploration has the effect of rotation, but the amount of rotation declines as the bandit begins to personalize. In this paper, we examine the contrarian assertion that perhaps rotation should be maintained to sustain novelty even after the multi-armed bandit is aware of which intervention is most effective for the user.

% % The challenges of static interventions, and the rising wave of personalization systems, call into focus: would a rotation strategy work? Or is it a weak palliative with little discernible effect? This led to our research question:

% \begin{resques}[RQ]
% Can a strategy of rotating interventions produce more effective behavior change systems?
% %Does rotating interventions is helpful in increasing effectiveness and lowering attrition in online behavior changes systems?}
% \end{resques}

%----------------------------
% HYP
%\section{Rotating Interventions}
%In this section, we review literature in behavior change systems and psychology to develop specific testable predictions regarding the research question.% Few studies has touched upon the quantification of effectiveness and attrition in the literature of online behavior changes systems. However, there are studies in behavior changes systems investigating the effectiveness of algorithmic interventions system. Meanwhile, previous work in behavior-change contexts has shown the porblem of attrition of behavioral interventions. This section presents a review of such studies and the proposal of our research hypotheses. 

% \jingyi{reviewed papers? i would just say past work}
%\subsection{Effectiveness over time}
%While behavior change systems can be effective~\cite{doi:10.1080/15228830802094429, Cuijpers2008, info:doi/10.2196/jmir.1376}, many review papers are more restrained in whether behavior change systems remain effective over long periods of time~\cite{doi:10.1111/j.1467-789X.2009.00646.x, 10.1371/journal.pmed.1000387, NORMAN2007336, 10.1007/978-3-319-07127-5_11}. The critique holds that behavior changes are long, complex processes, and the effectiveness of a system is hard to maintain indefinitely~\cite{prochaska1997transtheoretical}. Prior work suggests that the effectiveness of showing a static intervention cannot be maintained indefinitely~\cite{Hiniker:2016:MDE:2858036.2858403, riekert2013handbook}. For example, when a health behavior change system started sending email reminders, the first reminder was successful 28\% of the time, but by the fifth reminder it was successful only 18\% of the time~\cite{kaptein2015personalizing}. 

%A further meta-analysis of 88 computer-tailored interventions for health behavior change suggested that the efficacy of interventions decreases over time~\cite{krebs2010meta}. This prompts our first hypothesis: %These literatures prompt our first hypothesis:

%\begin{hyp}[H\ref*{hyp:decreaseovertime}] \label{hyp:decreaseovertime}
%Static interventions will suffer from decreased effectiveness over time.
%\end{hyp}

%\subsection{The impact of rotation}
%Novelty can be a driving factor for effectiveness. %Novelty has been studied for years in psychology and brain sciences. 
%One study showed that novelty can influence encoding of information into long-term memory, which, in turn, may raise awareness of behavioral changes~\cite{doi:10.1111/j.1467-9450.2005.00443.x}. Studies of gamification also explore the effect of novelty on user engagement~\cite{6758978}.  %Studies of gamification also explore the effect of novelty in supporting user engagement~\cite{6758978}. 

%In web design, people begin ignoring parts of the screen that have little information scent, such as ads. This phenomenon is termed banner blindness, after the commonness of the effect in internet banner advertising~\cite{benway1998banner}. As static interventions remain deployed, they may suffer from the same banner blindness and lack of novelty (wear-out) effects, suggesting a potential mechanism for the decreased effectiveness over time.

%Rotating interventions may counter these effects.
%Different interventions appear in different parts of the interface, making it less likely that the user would ignore them wholesale.
%In the system with static interventions, the wear-out effect may be one mechanism behind the decrement of effectiveness over time. On the other hand, 
%Online behavior change systems that use machine learning algorithms such as multi-armed bandits hone in on a small number of interventions to use~\cite{paredes2014poptherapy, rabbi2014automated}, but during the early exploration phases they are essentially rotating between interventions. %Evidence shows that behavior changes system which implemented the algorithm showed higher effectiveness in the short term~\cite{paredes2014poptherapy}. 
%Rotating between interventions without machine learning systems behind the scenes also has proven effective in PMS system. Comparing to PMS system that shows static intervention, 
%Systems that personalize interventions~\cite{kaptein2015personalizing} or deploy many micro-studies% Additionally, one study that examined \textit{Fitbit}\'s random rotation interventions, has found an increase in the effectiveness of such interventions
%~\cite{doi:10.1111/j.1740-9713.2015.00863.x}
%have generally found positive effects.

%Based on these results, non-static interventions may be effective. We hypothesize:

%\begin{hyp}[H\ref*{hyp:rotation}] \label{hyp:rotation}
%Rotation will increase effectiveness, compared to static interventions.
%\end{hyp}

%\textit{H2: Rotating interventions will increase effectiveness of a behavior change system, compared to always showing the same one.}

%\subsection{Attrition}

%Attrition is a major challenge in behavior change systems: a metastudy of eHealth interventions found that an attrition rate around 99\% over a 12-week period is normal~\cite{eysenbach2005law}. %Similarly, attrition may also exist in our online behavior change system with rotating interventions. Studies in behavior changes systems report attrition of users over time. 
%Likewise, the number of users in a stress-coping mobile application declined in a steady rate through the study~\cite{paredes2014poptherapy}. 

%Though rotating interventions aids novelty, the literature suggests that it may hurt attrition. Rotation violates usability heuristics such as consistency and user control~\cite{nielsen199510}. Specifically, users may perceive a loss of control when they are presented with ever-changing interventions, %As a result, users may treat the system as an authoritarian figure. This perception may 
%potentially leading to non-compliance behaviors and a higher attrition
%rate~\cite{coco2018hiniker}. %Furthermore, with interventions randomly genenerated by eletronic devices, persuasive systems are designed without a user-oriented perspective. 
%Typically, in attrition-risky domains such as education, an effective user-centered design is critical for minimizing attrition~\cite{Angelino2007learning}. In light of these results, we hypothesize:

%\begin{hyp}[H\ref*{hyp:attrition}] \label{hyp:attrition}
%Rotation will increase attrition, compared to static interventions.
%\end{hyp}



%----------------------------
% Below this line is
% Archived materials
%----------------------------

% Todo:

%----------------------------
% Opening

% The field of persuasive technology studies how technology can be used to influence behavior~\cite{Fogg:2002:PTU:764008.763957}. Persuasive technology systems have been successful in promoting behaviors such as sustainable resource usage~\cite{froehlich2009ubigreen}, fitness~\cite{consolvo2008activity}, sleep~\cite{kay2012lullaby,choe2011opportunities}, healthy eating~\cite{noronha2011platemate,epstein2016crumbs}, stress management~\cite{adams2014towards,sanches2010mind}, smoking cessation~\cite{paay2015personal}, and productivity~\cite{whittaker2016don, kim2016timeaware}. Computer tailored interventions - one of the most widely studied persuasive technology, are often implemented based on a static system (a single intervention)~\cite{stayfocusd, leechblock, selfcontrolapp, focusbooster} or on a dynamic system that keeps rotating new interventions in a designed sequence~\cite{paredes2014poptherapy, rabbi2014automated, Kaptein2013, Kaptein:2011:MBA:1978942.1978990}. However, we do not really understand the effectiveness of intervention over time nor the possible influences of rotating interventions on effectiveness and attrition of behavior change systems. Instead of treating the computer tailored behavior change systems as a black-box to increase effectiveness in a short-term, we aim for a deep understanding about the behavior changing systems and building a system that can maintain high effectiveness in a long-run, which are crucial for proving the efficacy of a behavior change system~\cite{Klasnja:2011:ETH:1978942.1979396}.

% %----------------------------
% \subsection{Learning the effectiveness and attrition}
% A number of behavior change systems have proven helpful in increasing the effectiveness of an intervention~\cite{krebs2010meta, kaptein2015personalizing}. They often involve rotating interventions algorithmically such as multi-armed bandits~\cite{paredes2014poptherapy, rabbi2014automated}.  However, a consequence of using multi-armed bandits algorithm is that there is a high rotation rate among interventions in the beginning (during exploration). During this process, it starts showing the same intervention with increasing probablity (during exploitation)~\cite{AUDIBERT20091876}. Prior work suggests that the effectiveness of showing a constant intervention over time cannot be maintained indefinitely~\cite{Hiniker:2016:MDE:2858036.2858403, riekert2013handbook}. Meanwhile, randomly rotating between new interventions on personal devices such as \textit{Fitbit} has found an increase the effectiveness of interventions~\cite{doi:10.1111/j.1740-9713.2015.00863.x}. 

% Furthermore, learning the attrition of behavior change systems is important. Attrition (or dropout) has been a problem among participants in health research~\cite{Eysenbach2005attrition}. For example, persuasive systems built for weight-control report have shown increased attrition rates in longitudinal studies~\cite{Bernier1986}. To the best of our best knowledge, although dropping-out of users from the study has been reported~\cite{paredes2014poptherapy}, the increments in attrition has not been discussed in detail or quantified.

% Currenty, studies in health are investigating in similiar concepts: Micro-Randomised Trial (MRT) and Just-in-Time Adaptive Interventions (JITAIs)~\cite{MRT2015Klasnja}. Contrary to traditional Randomized Control Trial (RCT) which randomizes all the participants in the beginning, MRT and JITAIs are experimental design conceptions that randomize particpants in a micro-steps, in which participants can be exposed to different interventions. Our findings about attrition of rotating interventions can potentially contribute to health science research.

% In summary, there is a need for taking the wearout effect among various interventions into account when quantifying the effectiveness of single intervention and rotating interventions. The results would greatly help with designing a better persuasive system for behavior changes. Additionally, research on attrition of interventions could provide useful insights into building effective behavior change systems in the long-run. Thus, we proposed our research questions in the following: (RQ1,2,3)

% %----------------------------
% \subsection{Combating with attrition}
% Learning to build a behavior change system that compensates for the attrition has become an imperative. Behavior changes are often long-term complex processes that require years of efforts to accomplish on one\'s will~\cite{prochaska1997transtheoretical}. Steady drop-out rate in intervention systems exists in previous studies of personal behavior change systems~\cite{paredes2014poptherapy, krebs2010meta}. Drop-outs will not only result in the termination of the intervention system but also in a decrease in efficacy of such system~\cite{Klasnja:2011:ETH:1978942.1979396}. Thus, lowering the attrition in the long-run becomes a crucial step for making an effective behavior change systems. However, there has been little research on improving existing online intervention systems to lower the attrition.

% Outside the field of persuasive technology, lowering the attrition has been \textit{de facto} studied for years. Known as the IKEA effect, studies have shown that participants tend to value more on self-made products~\cite{NORTON2012453}. The attrition might be decreased if the participants are actively contributed to building the intervention systems. Meanwhile, various methods have been proposed in minimizing attrition in longitudinal health research including incentives, reminders and follow-up interviews~\cite{BOYS2003363, RIBISL19961}. Using a combination of different retention strategies also appears to lower the attrition~\cite{ROBINSON20151481}. Surprisingly, participants also show steady engagement in studies if opt-in/opt-out option is provided~\cite{doi:10.1080/13645570701334084}. Framing the interventions is also found to be useful to enhance effectiveness of a behavior change system~\cite{kim2016timeaware}.

% Taken together, previous literature has painted a picture where lowering the attrition is necessary to designing an effective intervention system. In economics and health studies, numerous studies have shown ways to decrease attrition of participants over time. This observation motivates our research questions: (RQ4,5,6)



% \subsection{Interface Design and Users Behavior}

% H4: Users who enable or disable interventions during the beginning are less likely to attrition.
% Cite something in psyc literatures that prove IKEA effect, and demostrate root cause of it. [cite needed here.] 
% Talke about IKEA effect in a systematic way. Present some concrete results here, how the IKEA, "do-it-youself" help making more and more costumers engaged and love their products. [cite needed here.] 
% Cite any researches done in the HCI domain field, if any. Talk about how design the system in embed IKEA effect will reduct attrition. [cite needed here.]

% H5: Reminding them how system work will improve user's mental model and reduce attrition.
% Reminding system is everywhere, and they are here for a reason. Reminding system works, improve efficacy, show concrete results here[cite needed here.] 
% Mental model formation of users [cite needed here.]  How is mental model related to attrition in other literature [cite needed here.] 
% Given concrete examples when reminding system work in increasing user engagement and reduce drop out rate.

% H6: Users feel of control and attrition over time. Given them opt-out option will actually reduct attrition.
% Cite peoples natural inclination of control things [cite needed here.]. 
% Cite literatures where user opt-in/opt-out actually will reduce attrition in a long run[cite needed here.], in average. because when user opt-out, then should be count as a negtive instance. But in avrage, the hypothesis we are proposing is, the attrition will go down. 
% Cite literatures outside HCI, when user does not feel of control, they will drop out. The downside of rotating interventions quickly[cite needed here.]. 
% We can also cite previous researches show user dropout[cite needed here.]. 
% We suspect that it might be due to lack of control. Thus, we propose this hypothesis.

% In health fields, 

% Probably from one of the grant proposals or so

% this one shows existence of novelty effects \cite{krebs2010meta}

% https://www.sciencedirect.com/science/article/pii/S0091743510002318

% poptherapy

%Despite numerous previous research in computer-tailored interventions system using algorithms such as multi-armed bandits, the existing literatures mainly focus on the comparisons of the effectiveness between the algorithmic system and other non-algorithmic system such as showing static interventions or selecting randomly - there are few direct investigations discuss that effects of the algorithmic intervention system in a long-run. Users might feel fatigue of novelty - alternating between different interventions and feel lack of control which might lead to aversion to the algorithm. In our study, we analyze not only the effectiveness of computer-tailored interventions system but also the attrition rate of end users. Additionally, majority of previous works on computer-tailored interventions system focus on health issue and very few of them touches on on-line productivity.

%In health fields, Paredes et al created a computer-aid intervention system, which used multi-armed bandits algorithm to help people to cope better with stress\cite{paredes2014poptherapy}. They found that participants statistically significantly lowered the stress level after using their system. They also found that setting a hard constraint for exploration rate as 50\% and thus proposed that novelty would ensure the efficacy of their system. However, they did not consider the decline of effectiveness of intervention over time in any experiment. The attenuation of intervention effectiveness was proved in many previous studies\cite{krebs2010meta}. On the other hand, we found that constantly switching between interventions might have negative consequences. For instance, the users might be fatigued and feel lack of control if the users saw different interventions frequently.

%Rabbi et al developed a system that encouraged users for more healthy behavior changes by displaying messages in a mobile health app\cite{rabbi2014automated}. They used multi-armed bandits algorithm to personalize contextual suggestions and found the adaptive algorithm more actionable. However, they only deployed their research in small recruited groups as a short-term process in which the relation between the adaptive algorithm and attrition rate was not considered in a long-run setting. 

%Similarly, Kerbs et al wrote a meta-analysis study of 88 computer-tailored interventions that were designed for health behavior changes in four different groups: smoking cessation, physical activity, eating a healthy diet, and receiving regular mammography screening \cite{krebs2010meta}. They found clinical and statistical significances in effectiveness for each of four behaviors. They also found the efficacy of interventions decreased over time unless the interventions were dynamically changing, which introduced novelty. Furthermore, they showed that greater effects for the novelty effect were seen in the long-run as well. Although they rigorously proved the existence of novelty effect (algorithmically alternating between different interventions) in computer-aid intervention system in health studies, they did not consider the possible negative effects might be come with this alternation between interventions, which we found in our study. The point of measuring the attrition rate in a long-run was crucial for health behavior change in HCI research because it increase the validity of your research, as Klasanja et al pointed out that a longitudinal study was needed for proving the efficacy\cite{Klasnja:2011:ETH:1978942.1979396}.

%In technology fields, Kaptein and Markopoulos analyzed the importance of implicit profiling in various persuasive systems: Persuasive Messaging System(PMS) and Persuasion profiling in e-commerce\cite{kaptein2015personalizing}. They found that dynamically profiling users using the adaptive algorithm (multi-armed bandits algorithm) over-performed any other persuasive system, including pre-selecting a fixed persuasive system by educated crowds. We found similar trends as we discovered that alternating between different interventions could increase the effectiveness of our tool. However, Similar to previous works in health, they did not consider how the novelty might affect the attrition rate of end-users as well.
