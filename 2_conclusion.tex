\section{Conclusion}

%\zen{in the related work section, i did not mention isolated effect / symbiotic effect. will this confuse our readers?}
In this paper we have compared three hypotheses for how productivity interventions influence time spent on sites, apps, and devices other than the ones they are targeting. Productivity interventions may have no effect on other goals (\textit{isolated effects}), they may cause time to be redistributed to other unproductive goals (\textit{redistribution}), or they may cause a reduction in time spent on other unproductive goals (\textit{reduction}).

We adjudicated between these hypotheses by varying the frequency of productivity interventions on goals that users set in the HabitLab browser extension and mobile app. When interventions were more frequent, users spent less time on their goal sites and apps, showing that the productivity interventions were effective. We also defined a metric of intensity that captures frequency of interventions within device, and investigated the effects of varying intensity of interventions for other apps/sites, on time spent on an app/site. The result differed by device: on the browser we observed a global reduction effect, with time on non-goal sites decreasing with increasing intensity of interventions. However, on mobile we observed no effect. We believe these differences are caused by differing usage patterns and platform differences: websites drive traffic to other websites via hyperlinks, but mobile apps try to keep users remaining on the app. %a symbiotic effect, with time on an app decreasing with increasing intensity of other apps. We believe these differences are caused by differing usage patterns: interventions on mobile are causing users to turn off their phones, but interventions on the browser are simply causing users to go to other sites.

% When analyzing the effects of varying frequency of productivity interventions across devices, we observed no significant effect. We can attribute this to either our sample size being small \msb{I discount that explanation; cut it?}, or that in the cross-device case the isolated effects hypothesis truly holds \msb{also seems like throwing up our hands. either give a serious attempt at explanation here, or just don't try.}.

%\msb{this paragraph seems out of date now too; reduction doesn't seem like a bad unintended consequence:}
We have shown that while productivity interventions can sometimes have effects on usage of other, non-targeted sites and apps, they are often isolated in their effects. Hence, when designing for behavior change, while we should be careful about our measurements and the possibility of unintended side effects, in the context of productivity interventions it appears that targeting individual productivity goals does not cause substantial negative second-order effects. % Hence, when designing for behavior change, we need to be careful about our measurements and it is good to take a holistic approach if possible: the effects on a particular targeted app or site that we observe in isolation are not necessarily the net effect it is having on overall usage.

% \msb{this paragraph seems out of date now too; reduction doesn't seem like a bad unintended consequence:} We have shown that productivity interventions can have unintended side effects on usage of other sites and apps. Hence, when designing for behavior change, we need to be careful about our measurements and it is good to take a holistic approach if possible: the effects on a particular targeted app or site that we observe in isolation are not necessarily the net effect it is having on overall usage.

% Hence, we have shown that productivity interventions can have unintended side effects on usage of other sites and apps. Sometimes, they are detrimental and redistribute time to other sites, but other times they are symbiotic and help improve productivity outside the particular app it was originally targeting. Hence, when designing for behavior change, we need to be careful about our measurements and it is good to take a holistic approach if possible: the effects on a particular targeted app or site that we observe in isolation are not necessarily the net effect it is having on overall usage.

\section{Acknowledgements}

This work was supported by a Stanford Human-Centered Artificial Intelligence seed grant. We thank the many users who have used and contributed ideas and feedback to HabitLab.
