\beforepreface
\prefacesection{Abstract}
\msb{shouldn't this be an abstract, not a preface?} \geza{done}

Behavior change systems help people manage their time online. However, existing productivity systems have tended to assume a one-size-fits all solution, whereas there are many factors - novelty effects, attrition, influences from other apps and devices, and differences in individual motivation - that we must take into account. That said, these effects have been researched mostly in small-scale labs studies in domains other than online behavior change, so there is a large space of opportunities for studying how these effects manifest in real-world online behavior change contexts, and how to design better behavior change systems using these insights. \msb{needs one sentence detailing what the problem is here --- what is the gap that habitlab is filling in the next sentence? why does it exist? why haven't we been able to address the issues from the previous sentence?} \geza{done} In this thesis we present HabitLab, an in-the-wild experimentation platform we developed for conducting behavior change experiments, as well as a set of studies we run \msb{verb tense?} on the platform. HabitLab is a browser extension and \msb{assume the exact platform will be gone in 10yr and write your paper so that it's still relevant. so not android, perhaps mobile phone} \geza{done} mobile phone app with over 12,000 daily active, voluntary users, that users install to help them reduce time online and on their phones. It works by displaying one of 20+ interventions whenever they open an app or site they wish to spend less time on.

We use HabitLab as a large-scale experiment platform to understand behavior change. In our first set of studies, we investigate novelty effects of interventions, finding that compared to always showing the same intervention, a strategy of rotating between different interventions improves intervention effectiveness, but at the cost of increased attrition. This attrition is partly due to users being unfamiliar with rotating interventions, and improving users' mental models with a notice shown whenever a new intervention is shown is able to reduce this attrition. In our second set of studies, we investigate whether reducing time on one site or app by intensifying interventions influences time on other sites, apps, and devices. We find that on the browser, reducing time on one site reduces time spent elsewhere, but we do not observe the effect on mobile devices, and do not observe cross-device effects. In our third set of studies, we investigate users' motivation levels over time as indicated by the difficulty of interventions they select. We find that users initially overestimate how difficult of interventions they want, and their choices of difficulty progressively decline over time. Thus, we have found that online behavior change is a domain where incentives for users and researchers line up such that researchers can run large-scale in-the-wild experiments gaining ecologically valid insights about how behavioral psychology and economics theories play out in the real world, while users benefit from the more effective, scientifically informed behavior change systems that we can develop using these experiments and data.

\msb{needs a final thought about what's the big picture --- what did we learn here at a high level, and what are its implications for society?} \geza{done}

%This thesis tells you all you need to know about...
\prefacesection{Acknowledgments}

We thank those who have contributed code, designs, and ideas to HabitLab: Matthieu Rolfo, Sarah Sukardi, Matthew Mistele, Julie Ju Young Kim, Wenqin Chen, Radha Jain, James Carroll, Sara Valderrama, Catherine Xu, Esteban Rey, Lewin Cary, Carmelle Millar, Colin Gaffney, Swathi Iyer, Sarah Tieu, Danna Xue, Britni Olina Chau, Na He Jeon, Armando Banuelos, Kaylie Zhu, Brahm Capoor, Kimberly Te. We thank the many users who have used and contributed ideas and feedback to HabitLab.

This work was supported in part by the Stanford Cyber Initiative, the National Defense Science and Engineering Graduate Fellowship (NDSEG) Program, as well as a Stanford Human-Centered Artificial Intelligence seed grant \msb{weave this up higher with the funders}. \geza{done}

\afterpreface