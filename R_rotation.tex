%\msb{start with an orienting paragraph. what are we going to cover in this chapter, and why? and, what's the roadmap?}

%\msb{the general role of a RW chapter is to cover the related work that sets up the context of your thesis. So, material that is relevant across all your chapters should show up here. Material that is specific to a single paper (e.g., rotation) should be in its respective chapter} \geza{done}

%\msb{A section I don't see here, but should be here: a thorough survey of behavior change systems. What is the design space of such tools? What is the state of the art? How does HabitLab draw on those tools, and how does it innovate?} \geza{done}

In this chapter we will cover work in sociotechnical systems, psychology, and behavioral economics related to behavior change and persuasive technology. We will begin with theoretical frameworks and taxonomies about behavior change, discuss examples of behavior change systems, discuss various tools which behavior change systems make use of, and finally explore the causes of attrition and why behavior change systems fail.

%----------------------------
% RQ
\section{Behavior Change Theories}
%The field of persuasive technology studies how technology can be used to influence behavior~\cite{fogg2002persuasive}. Persuasive technology systems have been successful in promoting behaviors such as sustainable resource usage~\cite{froehlich2009ubigreen}, fitness~\cite{consolvo2008activity}, sleep~\cite{kay2012lullaby,choe2011opportunities}, healthy eating~\cite{noronha2011platemate,epstein2016crumbs}, stress management~\cite{adams2014towards,sanches2010mind}, smoking cessation~\cite{paay2015personal}, and productivity~\cite{whittaker2016don, kim2016timeaware}. One common framework of behavior change is the B=MAT model~\cite{fogg2002persuasive}, which states that desired behaviors result when motivation, ability, and a trigger (a call to action) are all present. Another framework of habit change is the habit loop~\cite{eyal2014hooked}, which tells us that designs can build habits via a repeated process of displaying a trigger, having the user take an action, providing a reward, and having the user invest in the system.

\subsubsection{Theoretical Frameworks for Behavior Change}

%\msb{I don't really see the point of section headers for so few paragraphs. instead, just use strong topic sentences} \geza{done}

The field of persuasive technology studies how technology can be used to influence behavior~\cite{fogg2002persuasive}.
 There are a number of theoretical frameworks describing behavior change systems. B=MAT is a popular framework of behavioral change~\cite{fogg2002persuasive}, which demonstrates that systems can focus on three elements---motivation, ability, and a trigger (a call to action)---to produce behavior change. The habit loop is another framework for building habits~\cite{eyal2014hooked}, stating that systems can build habits through an iterated process of displaying a trigger, prompting the user to take an action, giving out a reward, and helping the user to invest in the system.

\subsubsection{Taxonomies of Behavior Change}

A number of taxonomies characterize the design space of interventions, both general~\cite{michie2013behavior, behaviourchangewheel, abraham2008taxonomy, dolanmindspace} and domain-specific~\cite{hardeman2000interventions, west2010behavior}. Michie's behavior change taxonomy lists 93 techniques for behavior change, clustered according to the cognitive phenomenon they target~\cite{michie2013behavior}. Systems have investigated effects of these techniques individually, such as using ``cheating'' to support lapse management~\cite{agapie2016staying}, using different framings to present results~\cite{kim2016timeaware}, or setting goals and plans~\cite{agapie2016plansourcing}.

\section{Behavior Change Systems}

Persuasive technology systems have been successful in promoting behaviors such as sustainable resource usage~\cite{froehlich2009ubigreen}, fitness~\cite{consolvo2008activity}, sleep~\cite{kay2012lullaby,choe2011opportunities}, healthy eating~\cite{noronha2011platemate,epstein2016crumbs}, stress management~\cite{adams2014towards,sanches2010mind}, smoking cessation~\cite{paay2015personal}, and productivity~\cite{whittaker2016don, kim2016timeaware}.

They can operate on many different platforms, such as the web or mobile devices. Web-based systems promote a behavior change goals including classroom engagement~\cite{anderson2014engaging, anderson2013steering}, psychology therapy~\cite{doi:10.1080/15228830802094429} and healthy habits~\cite{cugelman2013gamification, lyons2014behavior}. In parallel, a number of studies focused on mobile-based interventions~\cite{paredes2014poptherapy, RILEY201567, FJELDSOE2009165, Whittaker09, info:doi/10.2196/mhealth.4160}. For instance, MyBehavior, a mobile phone app, was built to track physical activities of the users and to provide personalized suggestions that are tailored to the users' historical behavioral data~\cite{info:doi/10.2196/mhealth.4160}. Similarly, PopTherapy is a mobile phone app that studied micro-interventions for coping with stress~\cite{paredes2014poptherapy}.

\subsubsection{Sociotechnical Systems for Behavior Change}

People use a variety of sociotechnical systems to support behavior change, including forums~\cite{eysenbach2004health, chancellornorms}, social sharing~\cite{poirier2012social, Chung:2016:BNA:2818048.2819926, pina2017personal, Ko:2015:NGI:2675133.2675244}, personal informatics~\cite{li2010stage, Chung:2017:PTB:3025453.3025747}, and self-experimentation~\cite{Karkar:2017:TFS:3025453.3025480}. People use behavior change forums to gain social support~\cite{hong2012outcomes} -- meeting social needs such as approval and esteem~\cite{kaplan1977social}. They do so by providing users with information and advice~\cite{hong2012outcomes}, and establishing norms~\cite{chancellornorms}. They also facilitate social comparisons~\cite{davison2000talks} which influence behaviors, as social comparison theory states that users seek to bring their behaviors in line with norms~\cite{festinger1954theory}. Communities also help users find others with similar experiences~\cite{huh2014health} who can help them through the process of recovering and adapting to changes~\cite{newman2011s}. Social sharing~\cite{poirier2012social, richardson2010online} works by helping users receive support through social interactions, and encouraging accountability~\cite{epstein2015nobody}. Personal informatics support behavior change through stages of preparation, collection, integration, reflection, and action~\cite{li2010stage}. The theory of lived informatics~\cite{epstein2015lived} adds additional stages where users choose tracking tools, and alternate between lapsing and resuming their tracking behaviors. % HabitLab combines personal informatics and self-experimentation to support behavior change. Our study draws on lived informatics by evaluating whether rotating interventions is an effective strategy to combat lapses such as ignoring interventions or uninstalling.

% Of course, sociotechnical systems may also have deleterious effects --- communities may set unhealthy norms such as encouraging eating disorders~\cite{chancellornorms}, and social sharing can cause anxiety~\cite{munson2010happier}, impression management concerns~\cite{consolvo2014designing}, and make users unwilling to set goals~\cite{munson2015effects}.

% \msb{Can you pull this together into a summary of what we draw from this literature and how it influences HabitLab or our study?}

\subsubsection{Online Behavior Change}

One major topic inspiring our work is users' desires to curb or control their time spent on social media sites. People pressure themselves to, and often do, make efforts to reduce their time spent on social media sites such as Facebook and Twitter~\cite{Sleeper:2015:ILI:2675133.2675193,schoenebeck2014giving}. Yet this is difficult because users turn to social media to address their need to belong, the need for self-presentation, the need for self-esteem~\cite{nadkarni2012people}, the need for entertainment and gratification ~\cite{raacke2008myspace}, and self-affirmation ~\cite{toma2013self}. Whether social media use improves well-being is a complex question depending on the nature of the engagement ~\cite{uysal2013mediating, marche2012facebook, lin2015emotional, kim2011facebook, muise2009more, sagioglou2014facebook, tandoc2015facebook}, but thanks to instant gratification and sites' use of gamification~\cite{chou2015actionable, zichermann2011gamification, huotari2012defining} and behavior design techniques~\cite{fogg2002persuasive, eyal2014hooked} to drive engagement, users keep coming back to the point that some consider it an addiction~\cite{andreassen2012development, ryan2014uses, tang2016personality, turel2014examination}. %Furthermore, social media sites make heavy use of gamification to drive engagement on their sites~\cite{chou2015actionable, zichermann2011gamification, huotari2012defining}, and are cited as successful applications of behavior change theories~\cite{fogg2002persuasive, eyal2014hooked}.} %Furthermore, social media sites are designed to maximize user engagement, with features closely following gamification strategies~\cite{chou2015actionable} and behavior change theories~\cite{fogg2002persuasive, eyal2014hooked}.}

\section{Tools Used By Behavior Change Systems}

Researchers in behavioral economics and related fields have developed a number of tools which are useful for building behavior change systems. Gamification, which introduces triggers, investment, rewards, and game-like elements to motivate behavior change, and personalization are a pair of tools which are often used to boost the effectiveness of behavior change systems. Another one of these tools is \textit{choice architectures}, or ways to structure choices to influence behaviors.

\subsubsection{Gamification}

Much previous work has focused on gamification as an approach to design behavior change systems~\cite{deterding2011game}. Gamification has been shown to have positive effects on engagement and outcomes in behavior-change contexts such as promoting healthy habits~\cite{cugelman2013gamification, lyons2014behavior} and improving  educational engagement~\cite{anderson2013steering, anderson2014engaging}, though effectiveness varies depending on the context and design~\cite{6758978}. % Popular online services such as Facebook and LinkedIn make heavy use of gamification to drive engagement on their sites~\cite{chou2015actionable, zichermann2011gamification, huotari2012defining}.


\subsubsection{Personalization}

A recent trend in behavior change systems has been the concept of personalizing interventions. Such systems explore several possible strategies using techniques such as multi-armed bandits to find the intervention that is most effective for the user~\cite{paredes2014poptherapy, rabbi2014automated}. For example, PopTherapy demonstrated personalized messaging could be found through such techniques~\cite{paredes2014poptherapy}. Likewise, HeartSteps conducted tens or hundreds of micro-randomized trials on users~\cite{doi:10.1111/j.1740-9713.2015.00863.x}.  % When multi-armed bandits are just beginning to get feedback from a user, they will try out several different interventions to see what works. This exploration has the effect of rotation, but the amount of rotation declines as the bandit begins to personalize. %In this paper, we examine the contrarian assertion that perhaps rotation should be maintained to sustain novelty even after the multi-armed bandit is aware of which intervention is most effective for the user.

\subsubsection{Choice architectures}

The field of behavioral economics has developed a number of theoretical frameworks for how to present choices to influence people's choices, known as choice architectures~\cite{thaler1980toward, thaler2009nudge, johnson2012beyond}. One of the best-known choice architectures is defaults -- the choice made if the user does not make an active choice -- which work by exploiting the status-quo bias~\cite{samuelson1988status}. Defaults have been found to be effective in numerous behavior change contexts, including increasing organ donations~\cite{johnson2003defaults}, encouraging saving for retirement~\cite{cronqvist2004design, madrian2001power}, and influencing choice of insurance plans~\cite{johnson1993framing}. Other examples of effective, widely deployed choice architectures include limiting the number of choices~\cite{cronqvist2004design, kling2008misperception}, sorting choices~\cite{lynch2000wine}, grouping choices~\cite{fox2005subjective}, and simplifying choice attributes to be more easily interpretable~\cite{peters2009bringing, soll2013consumer}. Some choice architectures are designed explicitly for interactive behavior change contexts where users repeatedly make choices, such as Enhanced Active Choice, which provides users with choices while attempting to steer them towards the desired one~\cite{keller2011enhanced}.

%\subsection{Choice Architectures to Reduce Myopic Choices}

A number of choice architectures have been developed to combat our bias towards myopic choices, aversion to uncertainty, and lead us to choices that have better long-term outcomes~\cite{johnson2012beyond}. If the user is choosing between short-term benefits and longer-term benefits, one strategy is to make the long-term outcomes of the choice more salient in the short term~\cite{weber2007asymmetric, soman2005psychology}. In cases where there are a large number of uncertain outcomes and we wish to encourage satisficing -- that is, choosing an acceptable option sooner rather than waiting for a hypothetical future optimal choice -- we can focus attention on the second-best outcomes~\cite{shu2008future}. This strategy of satisficing can both lead users to decide faster, and leave them happier with their choices~\cite{iyengar2006doing}. We can also combat procrastination resulting from users' tendency to be overly optimistic about future opportunity windows by explicitly enforcing limited opportunity windows~\cite{o1998procrastination}.




% One of the most comprehensive investigations we
% identified is a recent meta-analysis of 85 studies by
% Webb et al. (2010) that found interventions that were
% strongly based in theory had greater impact than those
% that were not, and interventions that incorporated more
% behavior change techniques tended to have larger
% effects than interventions that incorporated fewer techniques

% Within the CSCW community, behavior change has been an active area of research. One major topic inspiring our work is users' desires to curb or control their time spent on social media sites. People pressure themselves to, and often do, make efforts to reduce their time spent on social media sites such as Facebook and Twitter~\cite{Sleeper:2015:ILI:2675133.2675193,schoenebeck2014giving}. This paper builds on this literature, contributing studies of how people might become more effective at this goal. Sociotechnical systems are also locations where people discuss behavior change~\cite{chancellornorms}, and find social support~\cite{Ko:2015:NGI:2675133.2675244, Chung:2017:PTB:3025453.3025747}. Behavior change often requires self-tracking, self-experimentation, and personal informatics~\cite{Karkar:2017:TFS:3025453.3025480}, leading to opportunities to share data and progress with trusted others~\cite{Chung:2016:BNA:2818048.2819926, pina2017personal}.


% datu2012does

% Social media is particularly addictive be

% \rev{expanded the related work discussion on sociotechnical theories. list the theories driving these papers. social support cscw theories. why people use forums for behavior change support what role do other people play in driving your behavior change. collective exercise. why are social media particularly hard to manage your behavior with why is social media addictive.}

% \rev{festinger's social comparison theory postulated that social behaviors could be predicted largely on the basis of the assumption that individuals seek to have and maintain a sense of normalcy and accuracy about the world}

% http://journals.sagepub.com/doi/pdf/10.1177/1524839911405850
% Harnessing Social Media for Health Promotion and Behavior Change

% \rev{Communities and collaborative support can help achieve behavior change goals. }




%Attrition is a major challenge in behavior change systems: a metastudy of eHealth interventions found that an attrition rate around 99\% over a 12-week period is normal~\cite{eysenbach2005law}. %Similarly, attrition may also exist in our online behavior change system with rotating interventions. Studies in behavior changes systems report attrition of users over time. 
%Likewise, the number of users in a stress-coping mobile application declined in a steady rate through the study~\cite{paredes2014poptherapy}. 


% The challenges of static interventions, and the rising wave of personalization systems, call into focus: would a rotation strategy work? Or is it a weak palliative with little discernible effect? This led to our research question:

%\begin{resques}[RQ]
%Can a strategy of rotating interventions produce more effective behavior change systems?
%Does rotating interventions is helpful in increasing effectiveness and lowering attrition in online behavior changes systems?}
%\end{resques}

%----------------------------
% HYP


%----------------------------
% Below this line is
% Archived materials
%----------------------------

% Todo:

%----------------------------
% Opening

% The field of persuasive technology studies how technology can be used to influence behavior~\cite{Fogg:2002:PTU:764008.763957}. Persuasive technology systems have been successful in promoting behaviors such as sustainable resource usage~\cite{froehlich2009ubigreen}, fitness~\cite{consolvo2008activity}, sleep~\cite{kay2012lullaby,choe2011opportunities}, healthy eating~\cite{noronha2011platemate,epstein2016crumbs}, stress management~\cite{adams2014towards,sanches2010mind}, smoking cessation~\cite{paay2015personal}, and productivity~\cite{whittaker2016don, kim2016timeaware}. Computer tailored interventions - one of the most widely studied persuasive technology, are often implemented based on a static system (a single intervention)~\cite{stayfocusd, leechblock, selfcontrolapp, focusbooster} or on a dynamic system that keeps rotating new interventions in a designed sequence~\cite{paredes2014poptherapy, rabbi2014automated, Kaptein2013, Kaptein:2011:MBA:1978942.1978990}. However, we do not really understand the effectiveness of intervention over time nor the possible influences of rotating interventions on effectiveness and attrition of behavior change systems. Instead of treating the computer tailored behavior change systems as a black-box to increase effectiveness in a short-term, we aim for a deep understanding about the behavior changing systems and building a system that can maintain high effectiveness in a long-run, which are crucial for proving the efficacy of a behavior change system~\cite{Klasnja:2011:ETH:1978942.1979396}.

% %----------------------------
% \subsection{Learning the effectiveness and attrition}
% A number of behavior change systems have proven helpful in increasing the effectiveness of an intervention~\cite{krebs2010meta, kaptein2015personalizing}. They often involve rotating interventions algorithmically such as multi-armed bandits~\cite{paredes2014poptherapy, rabbi2014automated}.  However, a consequence of using multi-armed bandits algorithm is that there is a high rotation rate among interventions in the beginning (during exploration). During this process, it starts showing the same intervention with increasing probablity (during exploitation)~\cite{AUDIBERT20091876}. Prior work suggests that the effectiveness of showing a constant intervention over time cannot be maintained indefinitely~\cite{Hiniker:2016:MDE:2858036.2858403, riekert2013handbook}. Meanwhile, randomly rotating between new interventions on personal devices such as \textit{Fitbit} has found an increase the effectiveness of interventions~\cite{doi:10.1111/j.1740-9713.2015.00863.x}. 

% Furthermore, learning the attrition of behavior change systems is important. Attrition (or dropout) has been a problem among participants in health research~\cite{Eysenbach2005attrition}. For example, persuasive systems built for weight-control report have shown increased attrition rates in longitudinal studies~\cite{Bernier1986}. To the best of our best knowledge, although dropping-out of users from the study has been reported~\cite{paredes2014poptherapy}, the increments in attrition has not been discussed in detail or quantified.

% Currenty, studies in health are investigating in similiar concepts: Micro-Randomised Trial (MRT) and Just-in-Time Adaptive Interventions (JITAIs)~\cite{MRT2015Klasnja}. Contrary to traditional Randomized Control Trial (RCT) which randomizes all the participants in the beginning, MRT and JITAIs are experimental design conceptions that randomize particpants in a micro-steps, in which participants can be exposed to different interventions. Our findings about attrition of rotating interventions can potentially contribute to health science research.

% In summary, there is a need for taking the wearout effect among various interventions into account when quantifying the effectiveness of single intervention and rotating interventions. The results would greatly help with designing a better persuasive system for behavior changes. Additionally, research on attrition of interventions could provide useful insights into building effective behavior change systems in the long-run. Thus, we proposed our research questions in the following: (RQ1,2,3)

% %----------------------------
% \subsection{Combating with attrition}
% Learning to build a behavior change system that compensates for the attrition has become an imperative. Behavior changes are often long-term complex processes that require years of efforts to accomplish on one\'s will~\cite{prochaska1997transtheoretical}. Steady drop-out rate in intervention systems exists in previous studies of personal behavior change systems~\cite{paredes2014poptherapy, krebs2010meta}. Drop-outs will not only result in the termination of the intervention system but also in a decrease in efficacy of such system~\cite{Klasnja:2011:ETH:1978942.1979396}. Thus, lowering the attrition in the long-run becomes a crucial step for making an effective behavior change systems. However, there has been little research on improving existing online intervention systems to lower the attrition.

% Outside the field of persuasive technology, lowering the attrition has been \textit{de facto} studied for years. Known as the IKEA effect, studies have shown that participants tend to value more on self-made products~\cite{NORTON2012453}. The attrition might be decreased if the participants are actively contributed to building the intervention systems. Meanwhile, various methods have been proposed in minimizing attrition in longitudinal health research including incentives, reminders and follow-up interviews~\cite{BOYS2003363, RIBISL19961}. Using a combination of different retention strategies also appears to lower the attrition~\cite{ROBINSON20151481}. Surprisingly, participants also show steady engagement in studies if opt-in/opt-out option is provided~\cite{doi:10.1080/13645570701334084}. Framing the interventions is also found to be useful to enhance effectiveness of a behavior change system~\cite{kim2016timeaware}.

% Taken together, previous literature has painted a picture where lowering the attrition is necessary to designing an effective intervention system. In economics and health studies, numerous studies have shown ways to decrease attrition of participants over time. This observation motivates our research questions: (RQ4,5,6)



% \subsection{Interface Design and Users Behavior}

% H4: Users who enable or disable interventions during the beginning are less likely to attrition.
% Cite something in psyc literatures that prove IKEA effect, and demostrate root cause of it. [cite needed here.] 
% Talke about IKEA effect in a systematic way. Present some concrete results here, how the IKEA, "do-it-youself" help making more and more costumers engaged and love their products. [cite needed here.] 
% Cite any researches done in the HCI domain field, if any. Talk about how design the system in embed IKEA effect will reduct attrition. [cite needed here.]

% H5: Reminding them how system work will improve user's mental model and reduce attrition.
% Reminding system is everywhere, and they are here for a reason. Reminding system works, improve efficacy, show concrete results here[cite needed here.] 
% Mental model formation of users [cite needed here.]  How is mental model related to attrition in other literature [cite needed here.] 
% Given concrete examples when reminding system work in increasing user engagement and reduce drop out rate.

% H6: Users feel of control and attrition over time. Given them opt-out option will actually reduct attrition.
% Cite peoples natural inclination of control things [cite needed here.]. 
% Cite literatures where user opt-in/opt-out actually will reduce attrition in a long run[cite needed here.], in average. because when user opt-out, then should be count as a negtive instance. But in avrage, the hypothesis we are proposing is, the attrition will go down. 
% Cite literatures outside HCI, when user does not feel of control, they will drop out. The downside of rotating interventions quickly[cite needed here.]. 
% We can also cite previous researches show user dropout[cite needed here.]. 
% We suspect that it might be due to lack of control. Thus, we propose this hypothesis.

% In health fields, 

% Probably from one of the grant proposals or so

% this one shows existence of novelty effects \cite{krebs2010meta}

% https://www.sciencedirect.com/science/article/pii/S0091743510002318

% poptherapy

%Despite numerous previous research in computer-tailored interventions system using algorithms such as multi-armed bandits, the existing literatures mainly focus on the comparisons of the effectiveness between the algorithmic system and other non-algorithmic system such as showing static interventions or selecting randomly - there are few direct investigations discuss that effects of the algorithmic intervention system in a long-run. Users might feel fatigue of novelty - alternating between different interventions and feel lack of control which might lead to aversion to the algorithm. In our study, we analyze not only the effectiveness of computer-tailored interventions system but also the attrition rate of end users. Additionally, majority of previous works on computer-tailored interventions system focus on health issue and very few of them touches on on-line productivity.

%In health fields, Paredes et al created a computer-aid intervention system, which used multi-armed bandits algorithm to help people to cope better with stress\cite{paredes2014poptherapy}. They found that participants statistically significantly lowered the stress level after using their system. They also found that setting a hard constraint for exploration rate as 50\% and thus proposed that novelty would ensure the efficacy of their system. However, they did not consider the decline of effectiveness of intervention over time in any experiment. The attenuation of intervention effectiveness was proved in many previous studies\cite{krebs2010meta}. On the other hand, we found that constantly switching between interventions might have negative consequences. For instance, the users might be fatigued and feel lack of control if the users saw different interventions frequently.

%Rabbi et al developed a system that encouraged users for more healthy behavior changes by displaying messages in a mobile health app\cite{rabbi2014automated}. They used multi-armed bandits algorithm to personalize contextual suggestions and found the adaptive algorithm more actionable. However, they only deployed their research in small recruited groups as a short-term process in which the relation between the adaptive algorithm and attrition rate was not considered in a long-run setting. 

%Similarly, Kerbs et al wrote a meta-analysis study of 88 computer-tailored interventions that were designed for health behavior changes in four different groups: smoking cessation, physical activity, eating a healthy diet, and receiving regular mammography screening \cite{krebs2010meta}. They found clinical and statistical significances in effectiveness for each of four behaviors. They also found the efficacy of interventions decreased over time unless the interventions were dynamically changing, which introduced novelty. Furthermore, they showed that greater effects for the novelty effect were seen in the long-run as well. Although they rigorously proved the existence of novelty effect (algorithmically alternating between different interventions) in computer-aid intervention system in health studies, they did not consider the possible negative effects might be come with this alternation between interventions, which we found in our study. The point of measuring the attrition rate in a long-run was crucial for health behavior change in HCI research because it increase the validity of your research, as Klasanja et al pointed out that a longitudinal study was needed for proving the efficacy\cite{Klasnja:2011:ETH:1978942.1979396}.

%In technology fields, Kaptein and Markopoulos analyzed the importance of implicit profiling in various persuasive systems: Persuasive Messaging System(PMS) and Persuasion profiling in e-commerce\cite{kaptein2015personalizing}. They found that dynamically profiling users using the adaptive algorithm (multi-armed bandits algorithm) over-performed any other persuasive system, including pre-selecting a fixed persuasive system by educated crowds. We found similar trends as we discovered that alternating between different interventions could increase the effectiveness of our tool. However, Similar to previous works in health, they did not consider how the novelty might affect the attrition rate of end-users as well.
