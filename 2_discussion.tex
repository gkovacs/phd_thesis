\section{Discussion}

We found that productivity interventions on the browser also reduced time on sites other than the targeted sites, but there was no such effect on mobile or cross-device.

We believe the reason we observed reduction in time on non-goal sites on the browser is several of the most popular goal sites --- such as Facebook, Reddit, Twitter --- are filled with hyperlinks to other sites, and hence drive traffic to them. For example, if an intervention makes a user spend less on their Facebook feed, they are going to be less likely to stumble upon a New York Times article, hence the Facebook-reducing intervention may also reduce time on New York Times.
Part of this may be a difference in how mobile applications work, compared to websites. Several mobile applications embed a web browser so that even if the user clicks a link, it will open within the same app. For example, Facebook is one such app, so if the user clicks on a New York Times link within the Facebook app, it is opened within the Facebook app's built-in browser, so the time they spend reading that article will still be counted towards Facebook app usage.

One possible reason for differences between mobile and web is that the apps users choose to reduce time on in each two platform differ (e.g., messaging apps on Android vs. link aggregators on Chrome). 
There also exist differences in typical interaction styles (short, notification-driven sessions on Android~\cite{oulasvirta2005interaction}, vs. longer sessions resulting from self-interruption on Chrome). 85\% of the apps that Android users frequently chose to reduce time on are for messaging (WhatsApp, Instagram, Facebook Messenger, Twitter, LINE, Snapchat), where a characteristic interaction is receiving a message, unlocking the phone to read it and reply, then turning off the screen (as shown in Figure~\ref{fig:android_sankey_v2}). Thus, users would not be drawn to other apps during this interaction. In contrast, with the Chrome version, the most selected sites are Facebook, YouTube, Reddit, and Twitter, 75\% of which are aggregators of links to other sites. The number of daily sessions per app is also greater on Android, though sessions are longer on average on Chrome, and stopping using the browser after a session ends occurs less on Chrome. Thus, the browser-based interactions users were using HabitLab to reduce are not short messaging-driven spurts that end with turning off the screen as on mobile, but rather long sessions of surfing through link aggregators ending with going to another site. So, a proposed mechanism: interventions short-circuit browsing long browser-based sessions, but mobile sessions are already short. % --- so users direct themselves to the browser, which begins a longer session.
% : comparing Facebook on both platforms, it is a median of 3 on Chrome, vs 9 on Android

This work brings about implications for designing interventions. Namely, we should consider not only the immediate interaction and its immediately measurable effects, but its longer-term effects in the context of the broader workflow. For example, consider 2 interventions for Facebook: 1) asks users to return to the home screen, vs 2) asks users to turn off the screen. Assuming similar rates of compliance, we would expect that measuring the effects on time spent on Facebook in isolation will show no difference between them. However, if we consider that going to the home screen can lead to users opening other apps, we might predict that a holistic measurement that includes effects on other apps as well will prefer 2) over 1).
Or if designing interventions to reduce snacking, should we: a) ask participants to not eat anything until their next meal, or b) give them gum instead? While calorie intake from the immediate interaction would favor a), b) may prevent future snacking down the line.
That said, in many cases, interventions are indeed isolated in their effects, and can even have beneficial effects elsewhere. % \msb{but don't our results also show that in many cases, things are isolated? I wrote that into the abstract given the results. whatever the takeaway is, I think it should be consistent in the intro+abstract+here.}

The findings of this chapter about time redistribution have a positive tone -- we did not observe negative side effects of productivity interventions, which would have been predicted if users were using their devices to replenish willpower when exhausted. Perhaps one speculative explanation is that in the context of device usage, diminished willpower results in the user opening a site or visiting an app, but actually spending time on the site or app does not replenish this willpower -- only the initial act of opening the site or app does. This may have interesting implications if it is true in other domains as well -- for instance, if the user is on a diet and has a craving for doughnuts, would an intervention preventing them from eating a doughnut also suppress cravings for other fattening foods as well? If they give in to the craving, would stopping them after the first bite leave their craving equally satisfied as if we let them eat the whole doughnut?


% Although our cross-device analysis showed no significant effects, we observed an insignificant trend cross-device suggesting the direction of predicted by our redistribution hypothesis. A possible explanation for a redistribution effect would be that. However, given that we observed

% This redistribution of time, even if it is slight, suggests that behavior change paradigms similar to HabitLab may be less effective and require greater vigilance than we hoped.


% Why would productivity interventions \jake{on Android} have a symbiotic effect, reducing time spent on other apps, but on the browser have a redistributive effect, increasing time spent on other unproductive sites? One possible explanation is the differing nature of the two devices. Mobile devices are primarily used for non-productive purposes and entertainment, which we use in many short sessions throughout the day -- the average user glances 100 times at a screen per day (todo actual number, need citation). Desktop and laptops, in contrast, are used for longer periods of time -- office workers will often spend nearly the entire workday in front of their computer, and with increasing amounts of productivity tools moving online, much of the workday may be spent on the browser. Hence, the effect of an intervention nudging the user off a site or app may differ across devices. We hypothesize that on mobile devices, the user's reaction may be to simply turn off the device, but on the desktop, they may instead go to a different unproductive site, or briefly switch back to a productivity application, but later go to a different unproductive site because their innate desire to procrastinate has not yet been fulfilled.

% Of course, if an innate desire to procrastinate is the underlying force driving the redistribution we observed on the browser, why didn't observe this effect on mobile? After all, even if an intervention causes the user to simply turn off their phone, do they not still have that unfulfilled desire to be unproductive? Perhaps the time is indeed being redistributed elsewhere to other unproductive activities, but the nature of mobile phones as devices that we carry around us everywhere, rather than being confined to the desk, makes it harder for us to observe where the time is going to. Perhaps the interventions on mobile are redistributing time to offline sources of procrastination, such as more water-cooler chats with colleagues. Without being able to track our users' every move, we cannot be sure whether this effect of mobile productivity interventions is truly symbiotic, or if it is simply redistributing to sources of unproductivity we cannot observe.

% Yet another possibility is the environment in which the devices are used. Perhaps the desktop browser is used primarily in an office setting, where the alternative to leaving Facebook is to go back to work, making Youtube seem to be a much more appealing place to go to if the user is feeling unproductive. On the other hand, perhaps users are using their phones in inherently recreational settings, such as in a mall or at a party, so if the user turns off their phone as a result of an intervention, they have plenty of other distractions in their environment to keep themselves happy and unproductive, and so they feel no need to go back to other apps on their phone.

