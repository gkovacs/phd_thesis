\section{Limitations}

Our methodology varied frequency of interventions, instead of comparing having interventions completely on vs completely off. This approach reduces the size of effects we can observe compared to having interventions completely on or completely off. Our approach is also sensitive to variance in the effectiveness levels of the interventions. Some interventions may be more aggressive than others and change users' behavior more drastically even with low frequency. This difference may alter time re-distributions due to varied frequency. % \msb{I don't know what difficulties of the interventions means in this context}.

We did not measure time spent on platforms that HabitLab does not support. For instance, HabitLab users may use Facebook on tablet devices, watch TV or engage in other activities that are considered unproductive aside from browsing on a desktop or on an Android phone. These behaviors may potentially change how time redistributed, but we are unable to track it. 

%There might be other long term effects that we did not observe, since the length of the deployment ($5.8$ weeks) is not long enough.
Additionally, our study explores time redistribution in the context of productivity. It is possible that this context may not generalize to other behavior change regimes.
% \zilin{done}
% \msb{Other limitations that could/should be mentioned:
% - untracked interactions. what if they checked facebook on a tablet or something else we couldn't see?\\
% - length of deployment (we may not see effects over extremely long periods.\\
% - sensitivity to particular interactions. we used frequency as a proxy for how aggressive habitlab is. but interventions differ in actual aggressiveness\\
% - we've only studied this in the context of productivity,  it's possible that these results might not generalize to other behavior change regimes
% }



